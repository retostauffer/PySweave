% -------------------------------------------------------------------
[latex header]
\documentclass[a4paper,10pt,headspline]{article}
\usepackage[utf8]{inputenc}
\usepackage[ngerman]{babel}
\usepackage[pdftex]{graphicx}
\usepackage{color}

\begin{document}
[/latex header]

[beamer header]
%% Begin slides template file
\documentclass[11pt,t,usepdftitle=false,aspectratio=169]{beamer}
\usetheme[nototalframenumber,foot,logo]{retostauffer}

\title[]{This is the tex source}
\subtitle{The {\LaTeX} Beamer Implementation}

\author[]{}
\URL{Reto.Stauffer@uibk.ac.at}

\footertext{{\LaTeX} beamer theme}
\date{2017-07-25}
\headerimage{1}

\begin{document}
[/beamer header]
% -------------------------------------------------------------------

\section{This is section one}

\begin{frame}[fragile]
\frametitle{This is a title}

\begin{doc}
This is text which should only be visible in the
documentation. Should be removed in the beamer output.

Das hier ist das Dokument in dem ich jetzt hoffentlich
hier ist kein Sweave stuff erlaubt.

Ausserdem sollten Sonderzeichen gehen, so dass ich äüö 
und so einen Mist schreiben kann.
\end{doc}

\begin{slides}
This is beamer!
\end{slides}

\end{frame}

\begin{frame}[fragile]
\frametitle{Das ist frame zwei}

\begin{doc}
Im Dokument sollte man nun nicht erkennen, dass hier
ein neuer Frame angefangen hat.
\end{doc}

\begin{slides}
Aber hier sind wir auf Frame zwei in den Slides.
\end{slides}

\end{frame}



% -------------------------------------------------------------------
[latex footer]
\end{document}
[/latex footer]

[beamer footer]
\end{document}
[/beamer footer]
% -------------------------------------------------------------------

